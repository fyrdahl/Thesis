% ==================================
% PACKAGES
% ==================================

% Make \cleardoublepage insert an empty page without a page number.
\usepackage{emptypage} 
\usepackage{blindtext}	% For "Lorem Ipsum" style place holders
\usepackage{graphicx}	% For \includegraphics
\usepackage{amsfonts}	% Math symbols
\usepackage{amsmath}	% Math symbols
\usepackage{amssymb}	% Math symbols
\usepackage{pdfpages}	% To include the PDFs of the papers
\usepackage{natbib}		% Reference in style of Natural Sciences
\usepackage{moresize}	% Additional font sizes \HUGE and \ssmal.
\usepackage{parskip}
\usepackage{multicol}
%\usepackage{pageslts}  % To calculate the total number of pages

% ==================================
% PAGE SIZE -- G5 OR E5
% ==================================

% G5 format (larger)
%\usepackage[paperwidth=169mm,paperheight=239mm,total={13.3cm,19.6cm}, top=1.8cm, ignorehead, centering, footskip=\footskip+4mm ]{geometry}

% E5 format (smaller)
\usepackage[paperwidth=155mm,paperheight=220mm,total={12cm,18cm}, top=1.8cm, ignorehead, centering, footskip=\footskip+4mm ]{geometry}


% ==================================
% CONFIGURE THE FONTS
% ==================================
\usepackage{ifxetex}
\usepackage{mathspec}
\usepackage[no-math]{fontspec}
\usepackage{xunicode}
\usepackage{xltxtra}
\AtBeginEnvironment{tabular}{\addfontfeatures{Numbers={Monospaced}}}

%
% (1) If the font is installed in your system:
%
%\setmainfont{Adobe Garamond Pro}
%\setsansfont{Frutiger LT Std 45 Light}
%\setmathfont(Greek,Digits,Latin){Adobe Garamond Pro}

%
% (2) If the fonts in the local directory.
%
%\setmainfont[
%  Ligatures=TeX,
%  Extension=.otf,
%  UprightFont=*-Regular,
%  BoldFont=*-Semibold,
%  ItalicFont=*-Italic,
%  BoldItalicFont=*-Italic,
%]{AGaramondPro}
%\setsansfont{[FrutigerLTStd-Light.otf]}
%\setmathfont(Greek,Digits,Latin){[AGaramondPro-Italic.otf]}

% mathspec is broken. The next eight lines work around that.
\usepackage{etoolbox}
\makeatletter
\begingroup\lccode`~=`"
\lowercase{\endgroup
  \everymath{\let~\eu@active@quote}
  \everydisplay{\let~\eu@active@quote}
}
\makeatother


% ==================================
% CHAPTERS, SECTIONS, SUBSECTIONS WITHOUT NUMBERS
% ==================================
% define commands for chapters, sections, and subsections
% that do not have a number, but are entered as candidates
% for the table of contents.
\newcommand\chap[1]{%
  \chapter*{#1}%
  \addcontentsline{toc}{chapter}{#1}
  \markboth{#1}{#1}
}
\newcommand\sect[1]{%
  \section*{#1}%
  \addcontentsline{toc}{section}{#1}
  \markright{#1}
}
\newcommand\subsect[1]{%
  \subsection*{#1}%
  \addcontentsline{toc}{subsection}{#1}
}


% ------------------------------------------------- %
% ------------------------------------------------- %
% ------------------------------------------------- %
% ------------------------------------------------- %

% Must come in the beginning. Changes the spacing in the table of contents to look more pleasing
%\usepackage{tocloft}
%\setlength{\cftbeforepartskip}{5.0mm}
%\setlength{\cftbeforechapskip}{2.0mm}
%\setlength{\cftbeforesecskip}{0.0mm}

% figure captions in bold (i.e. "Figure 1" in bold), sans serif, smaller font size, hanging label, always starting on the left side
%\usepackage{subfig}
%\DeclareCaptionFont{ssmall}{\ssmall}
%\DeclareCaptionFont{tiny}{\tiny}% "scriptsize" is defined by floatrow, "tiny" not
%\captionsetup{margin=0em,font={ssmall,sf},labelfont={bf},format=hang,singlelinecheck=false} 

% figures centred, smaller font in tables, captions on top for tables
%\usepackage{floatrow}
%\DeclareFloatFont{tiny}{\tiny}% "scriptsize" is defined by floatrow, "tiny" not
%\DeclareFloatFont{ssmall}{\ssmall}
%\floatsetup[table]{font={footnotesize},position=top}
%%%%%%%%%%%%%%%%%%%%%%%%%%%%%%%%%%%% end fonts %%%%%%%%%%%%%%%%%%%%%%%%%%%%%%%%%%%%%%%%%%%%%%%%%%%%%%%

% For tables spanning the full text width
%\usepackage{tabularx}

% For URLs use \url{<URL>}
%\usepackage{url}

% Chapters should have numbers - a typical thesis consists of two chapters:
% One to introduce and summarize the research ("kappa"), and one for
% reproductions of the papers and manuscripts. No need to number them by default.
% If you need chapter numbers back, comment the following line.
%\renewcommand{\thesection}{\arabic{section}} 

% Sections and subsections have numbers, subsubsections etc do not.
%\setcounter{secnumdepth}{2}

% Only chapters and sections appear in the table of contents, not subsections etc
%\addtocontents{toc}{\protect\setcounter{tocdepth}{1}}

% not every page needs to go to the same bottom line. Allows nicer page breaks.
%\raggedbottom

% avoid orphan/widow lines. Lower this number if necessary to get a good layout.
%\widowpenalty500
%\clubpenalty500



%%%%%%%%%%%%%%%%%%%%%%%%%%%%%%%%%%%%%%%%%%%%%%%%%%%%%%%%%%%%%%%%%%%%
% Define nice headers and footers
% To keep the thesis non-cluttered we only put the page number into
% the footer, and avoid headers
%\usepackage{fancyhdr}
%\fancyheadoffset{0cm}
%\pagestyle{plain}
%page number in the foot centre 
%\cfoot{\fancyplain{\thepage}{}}

% References should be a section of the summary text, with number and all ...
%\renewcommand\bibsection{\section{\bibname}\markright{\bibname}}