\chap{Introduction}
Non-communicable diseases kill 41 million people annually~\cite{WHO2013}. The biggest contributor is cardiovascular disease (CVD),\Nomenclature{CVD}{Cardiovascular disease} which kills 17.9 million people every year, accounting for more deaths than cancer and diabetes together. Early and accurate diagnosis of CVD is paramount for reducing mortality, and magnetic resonance imaging (MRI)\Nomenclature{MRI}{Magnetic Resonance Imaging} is a powerful diagnostic tool. However, compared to other modalities such as computed tomography (CT)\Nomenclature{CT}{Computed tomography} or echocardiography, MRI is still considered an inefficient method. With increasing demand and decreasing reimbursements in non-socialized healthcare systems, the need for highly efficient MRI is greater than ever before.

In contrast to other imaging modalities, MRI is characterized by excellent soft-tissue contrast, the absence of ionizing radiation, complete freedom in slice or volume placement. Furthermore, the signal in MRI is simultaneously dependent on multiple intrinsic properties of matter, providing endless variations for contrast manipulations.

Two major problems with MRI have yet to be solved. The first is the relatively long acquisition time compared to similar modalities such as computed tomography (CT), \Nomenclature{CT}{Computed tomography} single photon emission tomography (SPECT) \Nomenclature{SPECT}{Single photon emission tomography} or positron emission tomography (PET) \Nomenclature{PET}{Positron emission tomography}. The second is the inherent sensitivity to motion~\cite{Wood1985}. Radial imaging has been seen as a solution to both these problems, as it is inherently robust against motion~\cite{Glover1992}. Moreover, undersampling of a radial acquisition results in benign ``streak artifacts'' that are easily read through, meaning that the underlying structure is visible through the streaks. Moreover, radial imaging lends itself well to advanced reconstruction techniques, such as compressed sensing~\cite{Lustig2007}. All these properties make radial imaging a promising solution for highly efficient and robust imaging~\cite{Block2014}.

\sect{Disposition}

The ``Physiology'' chapter will offer a brief introduction to the field of cardiovascular physiology and introduce some concepts that will be necessary for the methodological discussion. The ``Magnetic Resonance'' chapter begins with an overview of the phenomenon of magnetic resonance and its applications in magnetic resonance imaging, followed by a brief introduction to some of the key methods used in this thesis. The ``Cardiovascular Magnetic Resonance Imaging'' chapter introduces some concepts of magnetic resonance imaging in the context of cardiovascular imaging. The ``Golden Angle'' chapter begins with a review of the math behind the golden angle and offers a review of methods using the golden angle, including the novel methods introduced in the thesis. The ``Materials and methods'' and ``Results'' chapters give an overview of the methods used in Studies I-IV that make up this thesis, and the main findings from each study. Finally, the ``Discussion'' and ``Conclusions'' chapters discuss the results in a broader context.