\chap{Conclusions}
In addition to utilizing golden-angle in novel applications, the thesis also explored methods for remedying problems that are usually associated with golden-angle radial sampling, such as eddy-current induced image artifacts and sampling non-uniformity when combined with physiological binning. The studies conducted as part of this thesis suggest that these problems can be overcome, making golden-angle imaging an attractive solution for cardiovascular MRI, especially for highly efficient high-temporal resolution imaging. This may push the boundary for the capabilities of cardiovascular MRI, or robust free-breathing applications, which can be of great benefit for under-served patient groups.

The specific conclusions for Studies I-IV were:
\begin{enumerate}[I.]
    \item The golden-angle radial trajectory can provide robustness to motion and flow when compared to a conventional Cartesian imaging method. The proposed method allowed for imaging the pulmonary vasculature with thin slices free from artifacts.
    \item Generalization of the generating function for the three-dimensional golden-angle profile ordering can be used to create golden-angle-like profile orderings with reduced angular steps between readouts, which leads to less eddy-current induced image artifacts and less sensitivity to off-resonance.
    \item The modified two-dimensional golden-angle method, SWIG, can be used in combination with a phase-contrast readout to enable high temporal resolution imaging of the diastolic function of the heart, with good agreement when compared to Doppler echocardiography.
    \item The modified, three-dimensional golden-angle method, 3D-SWIG, can improve sampling uniformity at high undersampling factors, which was demonstrated by performing three-dimensional free-breathing whole heart cine imaging in less than a minute.
\end{enumerate}

